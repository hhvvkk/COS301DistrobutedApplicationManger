\documentclass[a4paper,12pt,final]{article}


\usepackage{graphicx}
\title{
\begin{center}
  	\includegraphics[scale=0.3]{101Logo.png} 
  \end{center}
  \textbf{\\}
Integration Testing Documentation\\
}
\author{101 Solutions}

\begin{document}

\maketitle
\thispagestyle{empty}
\newpage
\tableofcontents
\thispagestyle{empty}
\newpage
\pagenumbering{arabic}

\section{Introduction}
This document provides an outline of the approach followed when doing integration testing. The actual system components are tested and will be tested across different levels of granularity.

\section{What we are using}
We are making use of QT unit tests which allows for a cross platform unit testing. This allows us to test the applicaiton for both windows and linux.






\section{Contract testing}
Within unit testing that are done there are testing done to determine if contracts are met. 
\begin{itemize}
\item If preconditions are met
\begin{itemize}
\item Service is provided
\item All post conditions hold true after service was provided
\end{itemize}
\item Any other units not directly tested will be mocked in order to simulate the unit requesting the service
\end{itemize}





\section{Unit Testing AppMan}
\subsection{Communication}

\subsection{Management Class}
\begin{itemize}
\item Signal Testing : Signals are tested in order to determine whether they exibit their required behavior when the management class should emit them.
\begin{itemize}
\item A slave machine is displayed on the widgets if another machine successfully connected
\item The slave that diconnects is removed off the window and only that machine which disconnected, not the others
\item The slave that diconnects is removed off the window
\item The application displays the percentage that a build is synched on the slave machine
\end{itemize}
\end{itemize}





\section{Unit Testing AppManClient}
\subsection{Communication}

\subsection{Management Class}
\begin{itemize}
\item Signal Testing : Signals are tested in order to determine whether they exibit their required behavior when the management class should emit them.
\begin{itemize}
\item The disconnect button shown if the management class emits that it has successfully connected
\end{itemize}
\end{itemize}





\end{document}